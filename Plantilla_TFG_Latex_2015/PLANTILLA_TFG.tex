%% Los margenes, tipo de hoja y estilo BOOK
\documentclass[a4paper,11pt,twoside,openright,titlepage]{book}
\usepackage[a4paper,left=1in,right=1in,top=0.6in]{geometry}

\usepackage[T1]{fontenc}    %Ineterprete de tíldes
%\usepackage[Latin1]{inputenc}
\usepackage{amsmath,amssymb}    %Paquete de entornos matematicos
\usepackage{listings}
\usepackage{natbib}
\usepackage[english,spanish]{babel}
\selectlanguage{spanish} 
\usepackage{graphicx}
\usepackage{psfrag}
\usepackage{float}
\usepackage{quotchap}
\usepackage{epsfig}
\usepackage[all]{xy}
\usepackage[utf8]{inputenc}
\usepackage{epsfig}
\usepackage{makeidx}
\usepackage{xcolor}
\usepackage{ifthen}
\usepackage{multicolpar}    %Para poner texto en columnas en plan articulo intercalado con texto normal
\usepackage{multicol,multirow}

\usepackage{url}        %Para direcciones web
\usepackage{marvosym}   %Para imprimir el simbolo de \EUR euro
%\usepackage{eurosym}   %Para imprimir el simbolo de \euro euro
\usepackage{fancybox}   %Para tablas con bordes redondeados


%% Modificación de la plantilla para adaptarla a los requisitos de PFC
\usepackage{fancyhdr}
\pagestyle{fancy}
%%% Cabeceras y pies de página
\fancyhead[CE,CO]{\emph{\titulo}}
\fancyhead[LE,LO,RE,RO]{}
\fancyfoot[LE,RO]{\thepage}
\fancyfoot[CE,CO]{\leftmark}

\renewcommand{\footrulewidth}{.6pt}


\colorlet{punct}{red!60!black}
\definecolor{background}{HTML}{EEEEEE}
\definecolor{delim}{RGB}{20,105,176}
\colorlet{numb}{magenta!60!black}

\lstdefinelanguage{json}{
    basicstyle=\normalfont\ttfamily,
    numbers=left,
    numberstyle=\scriptsize,
    stepnumber=1,
    numbersep=8pt,
    showstringspaces=false,
    breaklines=true,
    frame=lines,
    backgroundcolor=\color{background},
    literate=
     *{0}{{{\color{numb}0}}}{1}
      {1}{{{\color{numb}1}}}{1}
      {2}{{{\color{numb}2}}}{1}
      {3}{{{\color{numb}3}}}{1}
      {4}{{{\color{numb}4}}}{1}
      {5}{{{\color{numb}5}}}{1}
      {6}{{{\color{numb}6}}}{1}
      {7}{{{\color{numb}7}}}{1}
      {8}{{{\color{numb}8}}}{1}
      {9}{{{\color{numb}9}}}{1}
      {:}{{{\color{punct}{:}}}}{1}
      {,}{{{\color{punct}{,}}}}{1}
      {\{}{{{\color{delim}{\{}}}}{1}
      {\}}{{{\color{delim}{\}}}}}{1}
      {[}{{{\color{delim}{[}}}}{1}
      {]}{{{\color{delim}{]}}}}{1},
}

%Definiciones de funciones para los titulos
\newlength\salto
\setlength{\salto}{3.5ex plus 1ex minus .2ex}

\newlength\resalto
\setlength{\resalto}{2.3ex plus.2ex}

\newcommand{\lsection}[1]
                {\section{#1}
                \vskip-.9\resalto   %%%% Aquí reculo el posible salto por defecto de \section
                \hrule
                \vskip+.9\salto}  %%%% vuelvo ha realizar el salto (puedes poner otra vez el 90%)


%Para imágenes de entornos estáticos \captionFigure{Texto Caption}{Texto Label}
\newcommand{\captionFigure}[2]{
    \refstepcounter{figure}
    \centerline{Figura \thefigure: #1 \label{#2}}
    \addcontentsline{lof}{section}{\thefigure.\ #1\label{#2}}
}

%Para imágenes de entornos estáticos \NOcaptionFigure{Texto Caption}{Texto Label} "No escribe el caption"
\newcommand{\NOcaptionFigure}[2]{
    \refstepcounter{figure}
    \addcontentsline{lof}{figure}{\thefigure.\ #1\label{#2}}
}


%% Datos del PFC

\newcommand{\titulo}{1718\_072\_IC \- Diseño e implementación de un hub de control domótico}
\newcommand{\autor}{Autor: Pallarés Jiménez, Ignacio}
\newcommand{\director}{Nombre Apellido1 Apellido2}
\newcommand{\tutor}{Tutor: Delgado Mohatar, Óscar}
\newcommand{\ponente}{Ponente: Anguiano Rey, Eloy}
\newcommand{\vocal}{Nombre Apellido1 Apellido2}
\newcommand{\vocalsup}{Nombre Apellido1 Apellido2}
\newcommand{\presidente}{Nombre Apellido1 Apellido2}
\newcommand{\presidentesup}{Nombre Apellido1 Apellido2}
\newcommand{\fecha}{JULIO 2018}
\newcommand{\carrera}{Grado en Ingeniería Informática}

\begin{document}
\setlength{\baselineskip}{18pt}  %% Espacio interlinea
\setlength{\parskip}{6pt plus 1pt minus 1pt} %% Espacio interpárrafo
\setlength\itemsep{6pt plus 1pt minus 1pt}
\input{portada}

\frontmatter %Define el cuerpo inicial del libro en numeración con letras romanas

\input{primera_pag}

\chapter*{Resumen}

\section*{Resumen}

La domótica consiste en la automatización del hogar. Los sistemas domóticos, son aquellos capaces de domotizar una vivienda, y proporcionan servicios de comunicación,
seguridad, eficiencia energética...etc. Sin duda alguna, la comunicación entre estos sistemas es algo esencial, existiendo redes cableadas e inalámbricas para ello, pudiendo
ser controlados estos sistemas desde dentro y fuera del hogar.

BRIMO es un proyecto de código abierto para la gestión y el control de dispositivos domóticos en el hogar. Es una alternativa open source de bajo coste para todas
aquellas personas que deseen domotizar su hogar de una manera barata y sencilla. No utiliza protocolos privados, y cualquiera que lo desee puede utilizar y modificar
la aplicación a su gusto.

Además, cualquier persona puede crear sus dispositivos (sensores, actuadores o cámaras) de manera sencilla, siempre que éstos sigan los requisitos establecidos.
Brimo nos ayuda, gracias a una interfaz web sencilla e intuitiva, a ordenar nuestros dispositivos, visualizar sus estados y mandar comandos a los dispositivos que los acepten.

La aplicación está pensada para ejecutarse en entornos ligeros, concretamente en una Raspberry Pi 3 (precio asqequible), pero también puede ser ejecutada en cualquier ordenador tras una simple configuración.
Utiliza una arquitectura REST sobre el protocolo HTTP para comunicarse con los dispositivos, y una arquitectura MVC (Model View Controller) para la interacción con el usuario.

La función principal de Brimo es de "bridge", punto común entre el usuario y los dispositivos, se encarga de poner en contacto al usuario con los dispositivos.
\section*{Palabras Clave}
Domótica, código abierto, REST, raspberry, HTTP, sensores, MVC, actuadores, bridge.
\newpage

%-------------------------------------------------------------------------------------------------------------------------------------

\section*{Abstract}
Domotic consists of home automation. Domotic systems are those capable of automating a home, and provide communication services,
security, energy efficiency ... etc. Communication between these systems is essential, existing wired and wireless networks for it,
that allows us to controll them from inside and outside the home.

BRIMO is an open-source project for the management and control of domotic devices in the home.It is a low-cost open source alternative for all people
 who want to domotize their home in a cheap and simple way. It does not use private protocols, and anyone can use or modify the application on its preferences.

Besides, anyone is able to create its own devices (sensors, actuators or cameras) in a simple way following the application requirements. Brimo helps us, thanks to
a simple and intuitive web interface, to arrange our devices, seeing their status and to send them commands.

The application is designed to be runned on lightweight devices, specifically into a Raspberry Pi 3 (low cost), but it could be also runned on any computer after a simple
configuration. It uses REST architecture over HTTP protocol to communicate with devices and a MVC (Model View Controller) architecture for the interaction with users.

The main function of Brimo is to act as bridge, the common point between users and devices: Brimo is the responsible of the communication between them.
\section*{Key words}

Domotic, open-source, low cost, REST, raspberry, HTTP, sensors, MVC, actuators.

\input{agradecimientos}

\tableofcontents

\newpage \thispagestyle{empty} % Página vacía

\addcontentsline{toc}{chapter}{Índice de Figuras}    %Para que aparezca en el índice
\renewcommand{\listfigurename}{Índice de Figuras} 
\listoffigures

\newpage \thispagestyle{empty} % Página vacía

\addcontentsline{toc}{chapter}{Índice de Tablas}    %Para que aparezca en el índice
\renewcommand{\listtablename}{Índice de Tablas} 
\listoftables

\newpage \thispagestyle{empty} % Página vacía

\mainmatter %Define el cuerpo principal del libro numeración normal.

% \input{preambulo}

\chapter{Introducción} 
\label{chap:intro}

\vspace{-0.2cm}

\lsection{Motivación del proyecto}

Los sistemas domóticos por lo general utilizan una arquitectura centralizada: un controlador (bridge) es el encargado de enviar y recibir información de los dispositivos domóticos y las interfaces.
 Se utilizan sistemas centralizados debido a que abaratan mucho el coste de los dispositivos domóticos, así los dispositivos tienen poca electrónica y programación, y la responsabilidad principal
 reside en el bridge. Este enfoque tiene sentido cuando se trata de muchos dispositivos en un hogar, que es el caso ideal, si solo tuviésemos un sensor carecería de sentido tener un sensor y un bridge para manejarlo.

El problema principal que existe con los sistemas centralizados se encuentra en la \textbf{compatibilidad} entre dispositivos y bridges.
Por lo que he observado~\cite{article:EstadoDelArte}, todavía falta mucha estandarización en el
ámbito de la domótica: cada fabricante usa sus medios y protocolos haciendo incompatibles bridges y dispositivos. Además, estos dispositivos no suelen ser 
muy asequibles. Por lo tanto, nos encontramos ante la necesidad de comprar todos los dispositivos de una misma marca o tener muchos bridges, lo que nos obligaría a manejar cada
dispositivo desde su correspondiente bridge.

La domótica puede hacernos la vida en el hogar mucho más sencilla, ayudándonos a ahorrar tiempo y dinero que podremos invertir en otras cosas. Los hogares todavía están muy poco automatizados, y mi principal motivación ha sido acercar la domótica 
a las personas y aprender acerca de ella. Gracias a nuestro sistema manejamos todos los dispositvos a través de un solo bridge de manera sencilla y eficaz.

\newpage
\lsection{Objetivos y enfoque}

El objetivo último de nuestro proyecto es desarrollar un sistema que sea capaz de recibir y enviar información de dispositivos domóticos y sea capaz de interactuar con el cliente. 

Los \textbf{requisitos} que debe cumplir nuestro sistema son:
\begin{itemize}
  \item \underline{Ligero.} Un bridge no debería necesitar demasiada capacidad de procesamiento y de memoria, y es necesario que no sea muy costoso, por lo tanto, la ligereza es requisito indispensable.
  \item \underline{Compatibilidad.} Necesitamos que nuestro bridge no sea únicamente compatible con un tipo de sensor, o un modelo de cámara
  \item \underline{Interfaz sencilla y adaptable a cualquier dispositivo.} Necesitamos que la interfaz de nuestro bridge sea compatible con cualquier dispositivo sin perder funcinalidad.
  \item \underline{Seguridad.} La seguridad en la domótica es algo indispensable, confío en que el día de mañana incluso las cerraduras de nuestras casas serán automáticas, y no podemos dejar la responsabilidad de la seguridad
  de nuestra a casa a un sistema con vulnerabilidades de seguridad.
  \item \underline{Escalable.} Nuestro sistema ha de ser escalable y debemos pensar en todo momento en ampliaciones y trabajos futuros. La domótica evoluciona a pasos agigantados y podríamos añadir funcionalidades a nuestro 
  sistema practicamente a diario. No obstante, es necesario acotar firmemente los límites de nuestro proyecto para ceñirnos a las horas que corresponden a un TFG, aunque debemos tener muy en cuenta en todo momento trabajos futuros y ampliaciones.
  Además, debemos tener en cuenta la escalabilidad: domótica en un hospital, en una ciudad...etc.
\end{itemize}

\lsection{Metodología y plan de trabajo}

Otro ejemplo de imagen:
\begin{figure}[h]
  \centerline{
    \mbox{\includegraphics[width=3.00in]{images/logo_uam.eps}}
  }
  \caption{Ejemplo pie de figura 2}
  \label{fig:norm_Daugman}
\end{figure}

\newpage \thispagestyle{empty} % Página vacía 

\chapter{Estado del arte}
\label{chap:estadodelarte}

\lsection{Introducción}
La domótica, \textbf{\textit{domus}} (casa en latín) \textbf{\textit{autónomo}} (autogobernado en griego), se refiere a todos los sistemas
que son capaces de automatizar un hogar (o cualquier otro edificio). La automatización del hogar nos puede ayudar con tareas del día a día,
y tienen multitud de aplicaciones. Explicaremos algunas aplicaciones prácticas de la domótica en la siguiente sección, y se hará un breve
resumen de la domótica en la actualidad para terminar con el capítulo.

\lsection{Aplicaciones}

La domótica tiene infinidad de aplicaciones que nos pueden ayudar en nuestro hogar. Los ámbitos principales en los que la domótica nos puede
ser de gran ayuda se encuentran:

\begin{itemize}
\item\underline{Seguridad}: la domótica nos puede ayudar a mantener nuestro hogar seguro de diferentes maneras. Las alarmas, por ejemplo, automatizan 
la acción de llamar a la policía si un intruso es detectado por un sensor. Además, las alarmas antifuego son capaces de detectar humo y hacer sonar 
sirenas e incluso ayudar a la extinción del fuego mediante la expulsión de agua.
\item\underline{Eficiencia energética}: los sistemas domóticos nos pueden facilitar el ahorro energético en nuestros hogares. Es una de las aplicaciones
más utilizadas, sobre todo en grandes edificios (hoteles, fábricas, hospitales...), ya que el ahorro energético supone un ahorro económico. Como ejemplo
de esta aplicación existen: luces encendidas a través de sensores de movimiento, termostatos automatizados que regulan la temperatura en función de la
temperatura ambiente y la franja horaria, enchufes inteligentes que se encienden o se apagan según las necesidades pre programadas.
\item\underline{Comodidad}: la domótica puede ahorrarnos mucho tiempo automatizando acciones cotidianas. Ejemplos de estas acciones son: apagado de luces,
programación de alarmas, control multimedia de los dispositivos, etc.
\item\underline{Accesibilidad}: esta aplicación de la domótica ayuda a las personas con limitaciones a tener más autonomía y seguridad en su día a día. Ya
existen dispositivos como la teleasistencia de la Cruz Roja \cite{CruzRoja:teleasistencia}, que permiten a las personas avisar a emergencias tan sólo pulsando un botón.
Además, existen sistemas de vigilancia remota, timbres lumínicos para personas con problemas de audición...etc.
\end{itemize}

\lsection{La domótica en la actualidad}

Actualmente existen multitud de dispositivos domóticos, fabricantes y protocolos de comunicación. Cerraduras inteligentes, cámaras de seguridad, altavoces
termostatos son algunos de los dispositivos que nos podemos encontrar en la actualidad.
\par
Existen instalaciones domóticas de tres tipos:
\begin{itemize}
\item\underline{Cableadas}: los dispositivos se comunican entre ellos por cable. Una instalación cableada requiere un gran coste, y en edificios antiguos 
son menos factibles.
\item\underline{Inalámbricas}: los dispositivos se comunican entre sí utilizando medios inalámbricos como WiFi, Bluetooth o infrarrojos.
\item\underline{Mixtas}: se trata de una instalación en la que existen tanto dispositivos cableados como dispositivos inalámbricos. Debido a la diversidad
de medios y protocolos utilizados en estas instalaciones, se requiere un ``bridge`` que gestione todas estas conexiones.
\end{itemize}

Además, existen diferentes arquitecturas de sistemas domóticos:
\begin{itemize}
\item\underline{Centralizados}: en estas arquitecturas existe un dispositivo central, también conocido como \textbf{bridge}, que es el encargado
de comunicarse con el resto de dispositivos: actuadores y sensores. La ventaja de estas arquitecturas es que pueden albergar dispositivos que utilicen
diferentes protocolos y medios físicos de comunicación (instalación mixta). Un inconveniente de este tipo de arquitectura es la necesidad del
 propio bridge, suele suponer costes elevados y la rotura o desconexión del dispositivo central supone la caída total del sistema.
\item\underline{Descentralizados}: en esta arquitectura existen varios dispositivos centrales. Los dispositivos se conectan a estos dispsoitivos centrales, 
y pueden existir diferentes tipos de dispositivos centrales. Estos dispositivos son más complejos, pero también son más ampliables, ya que puede que si existe
un número elevado de dispositivos la arquitectura centralizada puede colapsarse.
\item\underline{Distribuidos}: en esta arquitectura todos los dispositivos son a su vez controladores. Son capaces de enviar la información al sistema y actuar 
en consecuencia a información recibida de otros dispositivos. Estos sistemas son más fiables, ya que la desconexión de un dispositivo no significa 
la caída del sistema, pero son más complejos y requieren más programación.
\end{itemize}



\chapter{Diseño del sistema}
\label{chap:disenosistema}
En este capítulo se describirá el diseño del sistema desarrollado. 
En la sección 3.1 se detallará la arquitectura del sistema global. 
En el apartado 3.2 se profundizará en la arquitectura interna del HUB.
\lsection{Necesidades del sistema}
En esta sección se analizarán las necesidades de nuestro sistema y de los dispositivos con los que nos comunicaremos.
\subsection{Necesidades de los dispositivos}
Antes de empezar a diseñar el sistema, elegir el protocolo que se utilizará y el medio físico por el que se comunicarán 
nuestros dispositivos, es necesario analizar los dispositivos que podrán conectarse a nuestro HUB, así como sus necesidades. 
Una vez determinados los requisitos del protocolo se estudiará el medio físico de comunicación.
\par
Los principales dispositivos domóticos que he encontrado son: sensores de temperatura, sensores de humedad, 
sensores de luz, sensores de movimiento, medidores de distancia, sensores de humo, sensores magnéticos, cámaras, 
bombillas, enchufes, termostatos, motores, aires acondicionados, interruptores y altavoces. Estos dispositivos
 pueden ser divididos en dos grupos: \textbf{sensores y actuadores}. 
\par
Los sensores solamente enviarán información a nuestro HUB (comunicación unidireccional), 
mientras que los actuadores, además de enviar el estado en el que se encuentran, recibirán mensajes con diferentes comandos (comunicación bidireccional).
\par
Agrupación de los dispositivos encontrados:

\begin{figure}[H]
\centering
\includegraphics[width=6.00in]{images/descripcion_dispositivos.png}
\caption{Tipos de dispositivos}
\label{fig:descripcion_dispositivos}
\end{figure}


Para la definición del protocolo dividiremos los dispositivos en tres tipos:
\begin{itemize}
\setlength\itemsep{6pt plus 1pt minus 1pt}
\item Tipo 1 (sensores): el hub sólo recibe información de los sensores. El hub no necesita saber qué tipo de información recoge (número decimal, SÍ/NO...etc), simplemente la actualiza y la muestra al usuario.
\item Tipo 2 (actuadores): estos dispositivos envían información al HUB y son capaces de recibir comandos del tipo: ON/OFF, +/-, número decimal...etc.
\item Tipo 3: cámaras IP. Este sensor recibirá un tratamiento especial debido a la necesidad de una comunicación constante y rápida.
\end{itemize}
\subsection{Necesidades del sistema}
Una vez analizadas las necesidades de los dispositivos podemos analizar las necesidades de nuestro sistema.
Para el desarrollo de nuestro sistema necesitaremos una arquitectura que nos permita:
\begin{itemize}
\setlength\itemsep{6pt plus 1pt minus 1pt}
\item \textbf{Comunicación bidireccional entre los dispositivos y el hub:} es necesario que el hub conozca información de los dispositivos, 
registre dispositivos y gestione dispositivos; así como también es necesario que los dispositivos puedan recibir comandos provenientes
del hub.
\item \textbf{Flexibilidad:} el hub debe permitir aceptar dispositivos con diferentes comandos, y no ceñirse sólo a un número cerrado de comandos (ON/OFF, +/-,...).
De esta manera cualquier actuador podrá conectarse al HUB, siempre y cuando los comandos sean registrados de manera correcta.
\item \textbf{Comunicación entre el hub y la interfaz:} será necesario que la información de los dispositivos y el estado de los mismos sea accesible
a través de la interfaz de usuario. Además el usuario debe ser capaz de gestionar los dispositivos y enviar comandos a los actuadores
a través de la interfaz.
\item \textbf{Seguridad en la comunicación:} es imprescindible que toda comunicación se realice de manera segura, de tal manera que nadie pueda modificar o
acceder a nuestra información.
\item \textbf{Escalabilidad:} aunque durante la realización de nuestro proyecto nos centraremos únicamente en la comunicación mediante protocolo HTTPS, 
es necesario diseñar un sistema escalable que el día de mañana pueda funcionar con diferentes protocolos y dispositivos.
\end{itemize}
\lsection{Casos de uso}
Para ayudarnos a diseñar nuestro software, es de gran utilidad un diagrama de casos de uso, en el que se describen
todas las acciones que el usuario puede llevar a cabo.
\par
Además, el diagrama de casos de uso será de gran utilidad a la hora de diseñar la interfaz de nuestra aplicación, ya que para que pueda 
darse un caso de uso es necesario que la interfaz lo contemple.
\subsection{Actores}
Debido a que el usuario podrá interactuar totalmente con el sistema y podrá llevar a cabo acciones irreversibles (como borrar un dispositivo), es necesario
que nuestro sistema cuente con funcinalidad para la gestión de usuarios basada en roles, de tal manera que los usuarios puedan llevar a cabo sólo las acciones
que su rol les permite.
\par
Distinguiremos tres roles de usuario, y por lo tanto, tres actores en nuestro sistema:
\begin{itemize}
\setlength\itemsep{6pt plus 1pt minus 1pt}
\item \textbf{Usuario lurker:} este usuario sólo es capaz de ver el estado actual de los diferentes dispositivos, así como su localización.
\item \textbf{Usuario común:} este usuario, además poder ver el estado y la localización de los dispositivos, puede enviar comandos a los actuadores y gestionar
los dispositivos: eliminarlos, modificar su localización...etc.
\item \textbf{Usuario administrador:} la única diferencia de este usuario con el usuario común es que este usuario es capaz de gestionar usuarios: dar de alta
nuevos usuarios, cambiar el rol de los usuarios y eliminar usuarios.
\end{itemize}
\subsection{Diagrama de casos de uso}

\begin{figure}[H]
\centering
\includegraphics[width=7.00in]{images/casos_uso.png}
\caption{Diagrama de casos de uso}
\label{fig:casos_uso}
\end{figure}

\lsection{Diagramas de secuencia}
Como hemos descrito anteriormente, en nuestro sistema participarán diferentes ``partes`` o subsistemas, por lo que es necesario definir la interacción entre cada
una de ellas. Así como en el diagrama de casos de uso hemos descrito cómo el usuario interactuará con el sistema, en los diagramas de secuencia describiremos
cómo interactuarán nuestros subsistemas entre sí.
\par
Debido a que en la mayoría de casos de uso solamente interactúan el HUB y la interfaz, se realizarán dos diagramas de casos de uso: uno común para todas las acciones
en las que participan HUB e interfaz, y otro con los casos de uso en los que es necesaria la interacción de todos nuestros subsistemas.
\subsection{Subsistemas}
Los subsistemas que participarán en los diagramas de secuencia son:
\begin{itemize}
\item\textbf{HUB}: es la parte central o bridge de nuestro sistema. Se encarga de comunicarse con los dispositivos y con la interfaz de usuario. Única ejecución, contiene también
el servidor de datos.
\item\textbf{Interfaz:} interfaz a través de la cual el usuario podrá interactuar con el HUB y los dispositivos. Se ejecuta en el dispositivo móvil del usuario, por lo 
que existirá una ejecución por cada usuario utilizando la aplicación.
\item\textbf{Dispositivos:} son los encargados de informar al usuario de su estado y, en el caso de los actuadores, ejecutar una acción a partir de un comando.
\end{itemize}
\subsection{Diagrama de secuencia Usuario-Interfaz-Hub}
Este flujo es el que seguirán todos los casos de uso a excepción de los casos \textbf{3.2 Enviar comando a dispositivo} y \textbf{4.1 Visualizar información dispositivo}.
Es el flujo habitual que siguen los frameworks web que utilizan el modelo MVC:
\par
\begin{figure}[H]
\centering
\includegraphics[width=6.00in]{images/diagrama_secuencia_1.png}
\caption{Diagrama de secuencia Usuario-Interfaz-Hub}
\label{fig:diagrama_secuencia_1}
\end{figure}

\subsection{Diagrama secuencia Usuario-Interfaz-Hub-Dispositivos}
En este diagrama se explican dos flujos: 
\begin{itemize}
\item{\textbf{4.1 Visualizar información dispositivo:}} la interfaz, por defecto, muestra la información actualizada de los dispositivos. Los dispositivos mandan su
información al HUB de manera constante, y la interfaz se encarga de pedir el estado de los dispositivos al HUB. 
\item{\textbf{3.2 Enviar comando a dispositivo:}} un usuario desea enviar un comando a un dispositivo. La interfaz envía el comando al HUB, que es el encargado de 
enviárselo al dispositivo.
\end{itemize}
\par
\begin{figure}[H]
\centering
\includegraphics[width=6.00in]{images/diagrama_secuencia_2.png}
\caption{Diagrama de secuencia Usuario-Interfaz-Hub-Dispositivos}
\label{fig:diagrama_secuencia_2}
\end{figure}

\lsection{Protocolo}
En esta sección se describirá el protocolo que se utilizará en las comunicaciones entre los dispositivos y el HUB.
\subsection{Protocolo HTTPS}
El protocolo elegido para la comunicación entre dispositivos, hub e interfaz es el protocolo HTTPS (Hypertext Transfer Protocol Secure).
\par
Este protocolo nos da la posibilidad de implementar una API REST consumible por parte de los dispositivos y por parte de la interfaz,
sin necesidad de utilizar diferentes protocolos en cada caso.
\par
Una API REST proporciona una interfaz entre diferentes sistemas que utilicen HTTP/S como medio comunicación. Las APIs REST están muy estandarizadas a día de hoy y nos dan la capacidad de separar lógica y funcionalidad entre cliente y servidor, y de ser capaces
de utilizar diferentes lenguajes y tecnologías para cada lado. Es decir, podemos tener un servidor escrito en Express.js (JavaScript), una interfaz 
gráfica utilizando Angular5 (TypeScript), y unos actuadores/sensores que utilicen CherryPy (Python).
\par
Además, utilizar HTTPS nos ofrece la posibilidad de crear un canal de comunicación cifrado, de manera que la información que circula en dicho 
canal no pueda ser descifrada por ningún intermediario ni se pueda sufrir un ataque Man-In-The-Middle*.
\lsection{Arquitectura del sistema}
En esta sección se describirá el diseño y la arquitectura del sistema de manera global, incluyendo dispositivos actuadores, 
dispositivos sensores y el propio HUB.
\subsection{Arquitectura del sistema}
Teniendo en cuenta las necesidades de nuestro sistema realizaremos una arquitectura similar a las arquitecturas de microservicios,
en la que el hub y los actuadores serán hosts de un servidor REST y serán capaces de recibir y procesar peticiones.
Esto requerirá n + 1 servidores REST, siendo n el número de dispositivos actuadores.
\par
El hub recibirá peticiones de parte de la interfaz de usuario y de los dispositivos, y lanzará peticiones a los actuadores.
Para ello se establecerán dos APIS que serán publicadas por el HUB y consumidas por la interfaz de usuario y los dispositivos:
\begin{itemize}
\setlength\itemsep{6pt plus 1pt minus 1pt}
\item \underline{Interface API:} será consumida por la interfaz de usuario. Se encargará de enviar la información de los dispositivos al usuario: número
 de dispositivos, información, localización, etc... Además, permitirá al usuario gestionar dispositivos y enviarles comandos.
\item \underline{Sensors API:} será consumida por actuadores y sensores por igual. Permitirá a los dispositivos darse de alta en el sistema y actualizar
periódicamente su información.
\end{itemize}
\par
Los actuadores, además de lanzar peticiones al hub para informar de su estado, deberán ser capaces de recibir peticiones del hub con
diferentes comandos. Para ello se establecerá otra API que todos los dispositivos deberán seguir, para que así el HUB consuma la misma API
en los diferentes dispositivos. Será denominada en adelante como \underline{Actuators API}. Estableceremos y explicaremos estas tres APIs en el capítulo
\textbf{3.3.3 APIS}.
\par
Como mencionamos anteriormente, la seguridad es un requisito indispensable de nuestro sistema, y debemos asegurar que ningún intruso pueda acceder y/o modificar nuestra información.
Para cumplir nuestro requisito, todas las peticiones HTTPS serán securizadas, asegurando así una comunicación segura entre los diferentes servidores.

Esquema de la arquitecura a seguir:
\begin{figure}[H]
\centering
\includegraphics[width=6.00in]{images/esquema_arquitectura.png}
\caption{Esquema de la arquitectura}
\label{fig:esquema_arquitectura}
\end{figure}

\subsection{Modelo de datos}
En esta sección se describirán los modelos de datos a utilizar. Todos los datos residirán en el HUB, que será el encargado de orquestarlos,
organizarlos y mantenerlos.
\par
Analizando las necesidades del sistema nos encontramos con cuatro entidades a definir:
\begin{itemize}
\setlength\itemsep{6pt plus 1pt minus 1pt}
\item \underline{Dispositivos}: esta entidad se utilizará para almacenar la información de los actuadores/sensores. En el caso de los actuadores
será necesario guardar su dirección IP para poder enviarles comandos.
\item \underline{Comandos}: entidad para almacenar los comandos de los diferentes dispositivos. Cada comando tendrá un código y una descripción.
\item \underline{Habitaciones}: esta entidad nace de la necesidad de organizar los dispositivos de una casa en grupos más pequeños. Una manera lógica 
y muy común es por habitaciones, cada dispositivo podrá o no pertenecer a una habitación.
\item \underline{Usuarios}: es necesario restringir los usuarios que pueden tener acceso a nuestro sistema. Las credenciales de cada usuario se guardarán en esta tabla,
así como su rol.
\end{itemize}
Diagrama entidad-relación modelo de datos:
\begin{figure}[H]
\centering
\includegraphics[width=6.00in]{images/er_brimo.png}
\caption{Diagrama ER}
\label{fig:diagrama-er}
\end{figure}
\subsection{APIS}
Una vez numeradas las diferentes APIS a utilizar y definido el modelo de datos podemos comenzar a definir detalladamente cada una de las APIS.
Al tratarse de APIS REST, en la definición
de cada método es necesario informar: ruta del método, verbo (GET, POST, PUT, PATCH, DELETE), cuerpo de la petición (si existiera) y variables
de la ruta (si existieran).
\par
A no ser que se indique lo contrario, el cuerpo de todas las peticiones debe estar en formato JSON, y debe ser
informada la cabecera \textbf{Content-Type} con valor \textbf{application/json}.
\par
Como se ha descrito anteriormente todas las peticiones deberán ir securizadas, para lo que se utilizarán tokens JWT. Es imprescindible que el
token vaya en la cabecera \textbf{x-access-token} de cada petición, o de lo contrario la petición será denegada. La generación de tokens y la gestión
de usuarios es descrita por la API de login.
\par
Estas APIS funcionan como contratos entre publicador y consumidor, y es imprescindible que ambas partes consuman y publiquen de la 
manera acordada para que el sistema completo funcione. El cambio de uno de estos contratos debe ser indicado a todas las partes para que se tenga
en cuenta en los desarrollos futuros.
\par
Todos los métodos de estas APIs están enumerados y definidos en un proyecto de Postman desde el cual se pueden probar.
\par
\subsubsection{Sensors API}
Esta API será consumida tanto por sensores como por actuadores, y les permitirá registrarse en el sistema y actualizar su información.

\begin{itemize}
\setlength\itemsep{6pt plus 1pt minus 1pt}
\item \textbf{POST /brimo/sensors-api/devices}: se utilizará para el registro de dispositivos. En el cuerpo de la petición se informarán
 un nombre descriptivo (podrá ser modificado por el usuario más adelante) y frecuencia
de actualización de la información \footnote{ Si el dispositivo pasa más de los segundos informados sin actualizar información, entonces
el HUB lo considerará desconectado.}. Opcionalmente, en el caso de ser un actuador, el dispositivo informará de los comandos que es capaz
de recibir. Estos comandos vendrán en forma de array y deben contener descripción y código de comando.
Si el registro es correcto el HUB responderá con un 201 CREATED y un id de dispositivo. Este id será utilizado por el dispositivo más adelante
para enviar información al HUB.
\item \textbf{PUT /brimo/sensors-api/devices/\{device-id\}/info}: se utilizará para actualizar la información del dispositivo. El parámetro
device-id indicará el id del dispositivo, proveniente del registro. Si la información se actualiza correctamente el HUB devolverá 200 OK. Una vez
actualizada la información del dispositivo se actualizará la hora de última actualización \textbf{(lastupdate)}.
\end{itemize}

\subsubsection{Actuators API}
Esta API será consumida por el HUB, y permitirá al HUB enviar comandos a los dispositivos.

\begin{itemize}
\setlength\itemsep{6pt plus 1pt minus 1pt}

\item \textbf{POST /brimo/actuators-api/commands?command\_code=ON}: es el único método de la API. El HUB enviará esta petición para enviar
comandos al dispositivo. El código de comando debe haber sido informado previamente en la fase de registro del actuador.

\end{itemize}

\subsubsection{Interface API}
Como hemos explicado anteriormente, esta API será consumida por la interfaz de usuario y permitirá al usuario obtener información de los dispositivos,
gestionarlos y mandarles comandos:

\begin{itemize}
\setlength\itemsep{6pt plus 1pt minus 1pt}
\item \textbf{GET /brimo/interface-api/devices}: esta petición nos devolverá información de todos los dispositivos
dados de alta en el sistema. Al tratarse de una lista no se poblarán todos los campos del dispositivo, sólo los comunes: id del dispositivo, nombre,
frecuencia, fecha de última actualización de la información, id de habitación y descripción de la habitación.
\item \textbf{GET /brimo/interface-api/devices/\{device-id\}}: a diferencia de la petición anterior, se obtiene únicamente la información
del dispositivo indicado con el parámetro device-id. Esta información es más completa, y además de la información de la petición anterior
se obtiene la lista de comandos que acepta el dispositivo y su IP.
\item \textbf{DELETE /brimo/interface-api/devices/\{device-id\}}: a través de esta petición el usuario podrá eliminar el dispositivo indicado. A partir
de este momento el sistema denegará al dispositivo la comunicación con el mismo.
\item \textbf{PATCH /brimo/interface-api/devices/\{device-id\}/?room-id=12\&name=sensor-habitacion}: esta petición permitirá editar la habitación
en la que se encuentra el dispositivo y su nombre. Ambos parámetros room-id y name son opcionales, aunque al menos uno debe estar presente.
\item \textbf{DELETE /brimo/interface-api/devices/\{device-id\}}: a través de esta petición el usuario podrá eliminar el dispositivo indicado. A partir
de este momento el sistema denegará al dispositivo la comunicación con el mismo.
\item \textbf{PATCH /brimo/interface-api/devices/\{device-id\}/?room-id=12\&name=sensor-habitacion}: esta petición permitirá editar la habitación
en la que se encuentra el dispositivo y su nombre. Ambos parámetros room-id y name son opcionales, aunque al menos uno debe estar presente.
\item \textbf{POST /brimo/interface-api/devices/\{device-id\}/commands?command-code=ON}: se utilizará para enviar comandos al dispositivo informado.
La petición irá al HUB, que será el encargado de enviar otra solicitud al dispositivo correspondiente. Para ello, utilizará la IP del dispositivo.
\item \textbf{GET /brimo/interface-api/devices/rooms}: devolverá la lista actual de habtiaciones registradas. Se devolverán en forma de 
array y en cada una de ellas vendrán informadas descripción e identificador.
\item \textbf{POST /brimo/interface-api/devices/rooms}: se utilizará por el usuario para añadir habitaciones. Únicamente es necesario informar
el nombre de la nueva habitación. Si el registro de la habitación es correcto, entonces el HUB devolverá 201 CREATED con el id de la nueva habitación.

\end{itemize}


\subsubsection{Login API}
Esta API será utilizada para la generación de los JWT, que necesariamente, deben ir informados en las cabeceras de cada petición. Además, gestionará
los usuarios con acceso al sistema.
\begin{itemize}
\setlength\itemsep{6pt plus 1pt minus 1pt}
\item \textbf{POST /brimo/login-api}: se deberán informar los campos usuario y contraseña para la correcta generación del token. En el caso
de introducir credenciales inválidas el HUB devolverá 401 Unauthorized. En caso de éxito el HUB devolverá el token generado, que será válido
para las siguientes dos horas.
\item \textbf{POST /brimo/login-api/users}: servirá para añadir nuevos usuarios al sistema. Deberán informarse nombre de usuario y contraseña.
\item \textbf{PUT /brimo/login-api/users}: servirá para modificar la contraseña y/o el nombre de usuario actuales. Deberán ir informados en el
cuerpo de la petición al menos uno de los dos parámetros.
\end{itemize}

\lsection{Diseño de la interfaz gráfica}

Una vez analizados los requisitos de nuestro sistema y diseñado los flujos de datos, podemos pasar a diseñar nuestra interfaz gráfica. Antes de
enumerar las pantallas que tendrá nuestra aplicación y definir el diagrama de actividades, debemos enumerar los requisitos funcionales 

\lsection{Arquitectura del hub}
En esta sección definiremos los módulos que nuestro HUB deberá tener y las funciones que cada uno debe cumplir. La organización por módulos, además
de permitirnos organizar nuestro software de una manera clara, nos ayudará a añadir nuevos módulos y funcionalidad el día de mañana con facilidad.
\par
Por ejemplo, en el módulo de servicios residirá toda la lógica interna, mientras que el módulo enrutador será el encargado de ``traducir`` los datos
provenientes de la red a un modelo de datos conocido e invocar a los diferentes servicios. De esta forma, si en un trabajo futuro queremos añadir
dispositivos bluetooth crearemos un módulo bluetooth que reutilice nuestros servicios.
\par
Otro ejemplo sería la migración de nuestra base de datos a otro motor diferente; sólo necesitaríamos cambiar el módulo repositorio, el resto del sistema
se mantendría intacto.
\par
Cada módulo
debe ser independiente del resto, y la modificación interna de un módulo no debería requerir la modificación del resto de módulos.
\subsection{Módulo enrutador}
Este módulo será el encargado de gestionar las conexiones entrantes y de manejar la información proveniente del exterior. Para nuestro caso, que utilizaremos
el protocolo HTTPS, en este módulo residirán las implementaciones de las APIS anteriormente definidas. Se encargará de implementar todas las rutas, encapsular
los diferentes parámetros en objetos de nuestro modelo y enviar las respuestas y códigos necesarios tras la invocación al módulo de servicios.
\subsection{Módulo middleware}
A pesar de haber separado este módulo del módulo enrutador, este módulo está totalmente ligado a la utilización del protocolo HTTPS. 
Se trata de un módulo totalmente independiente del módulo enrutador, y tendrá dos funciones principales:
\begin{itemize}
\setlength\itemsep{6pt plus 1pt minus 1pt}
\item Interceptar todas las peticiones antes de que lleguen al enrutador y validar las cabeceras y el token JWT. Si el token no es válido entonces
se envía un 401 Unauthorized sin llegar al enrutador.
\item Interceptar los errores que se provoquen durante la ejecución del programa (independientemente del módulo) y traducirlos a respuestas HTTPS. Para esto 
será necesario utilizar un modelo común de error que pueda ser interceptado por este módulo.
\end{itemize}
\subsection{Módulo de servicios}
En este módulo residirá la totalidad de nuestra lógica de negocio. A este módulo ya llegan objetos modelados con nuestro modelo de datos, y es totalmente
independiente del protocolo utilizado. Se encargará de hacer llamadas a los repositorios correspondientes y de aplicar la lógica correspondiente.
\par
Un ejemplo de lógica sería el registro de dispositivos; una vez recibido un dispositivo y sus correspondientes comandos, el servicio se encargará de hacer
las comprobaciones correspondientes y guardar el dispositivo y después sus comandos.
\par
Además, el módulo de servicios transformará los posibles errores provenientes de los repositorios para encapsularlos en errores internos. Un ejemplo sería
transformar un error \textbf{``14 SQLITE\_CANT\_OPEN``} en el siguiente error: \textbf{``Error 01: no se ha podido acceder a la base de datos sqlite``}.
\par
Tanto la entrada como la salida de datos de los métodos de nuestros servicios seguirán el modelo de datos del HUB.
\subsection{Módulo repositorio}
Este módulo contendrá toda la gestión de los datos del HUB. Será invocado por el módulo de servicios, y 
será el encargado de gestionar las conexiones con la base de datos e insertar/obtener datos de la misma.
Este módulo recibe datos modelados con nuestro modelo de datos, pero no necesariamente la manera de enviarlos/guardarlos tiene que coincidir con nuestro modelo 
de datos. Sin embargo, el retorno de los métodos de este módulo si serán datos modelados.
\par
Si en un futuro se realizasen llamadas a terceros, una API de Google por ejemplo, las llamadas a esas APIS también se realizarían desde este módulo.
\subsection{Vista general}
Por lo tanto, el diseño esquemático de la arquitectura interna del hub sería el siguiente:
\begin{figure}[H]
\centering
\includegraphics[width=6.00in]{images/arquitectura_hub.png}
\caption{Arquitectura interna del hub}
\label{fig:arquitectura_hub}
\end{figure}


\chapter{Desarrollo del sistema}
En este capítulo se explicarán las tecnologías utilizadas y las arquitecturas internas de cada subsistema. Los dos subsistemas que desarrollaremos
serán el HUB y la Interfaz.
\label{chap:desarrollosistema}
\lsection{Stack tecnológico}
Elegir el stack tecnológico a utilizar en el desarrollo de cualquier sistema es siempre una tarea difícil y determinante en el desarrollo de 
cualquier proyecto. Una elección inacertada puede significar el fracaso de un proyecto, mientras que una elección acertada sginificará que el 
proyecto salga adelante cumpliendo con todas las expectativas.
\par
Para elegir el stack correcto necesitamos tener en cuenta varios aspectos:
\begin{itemize}
\item\underline{Madurez de la tecnología:} es importante utilizar una tecnología con cierta madurez para evitar bugs, incompatibilidades, etc. Además, es
conveniente optar siempre por versiones LTS y evitar versiones Beta.
\item\underline{Comunidad/respaldo de la tecnología:} por lo general, este punto está muy relacionado con el punto anterior, ya que cuanta más madurez tiene
una tecnología, más comunidad suele existir. Aunque no sólo la madurez influye, si no la popularidad de la tecnología, cuanto más se use esa tecnología,
más comunidad tiene. La comunidad de una tecnología, junto con su documentación, ayudan al desarrollador a enfrentarse a los problemas que surgen a lo
largo del desarrollo, siendo capaces de ver o preguntar a otros desarrolladores que ya se hayan enfrentado a dicho problema.
\item\underline{Conocimiento de los desarrolladores:} este punto es esencial, ya que es necesario que el equipo de desarrollo conozca las tecnologías, o en caso
contrario, ser capaces de contratar nuevos desarrolladores que sí las conozcan. La popularidad de la tecnología es esencial en este punto, ya que, por 
lo general, es más sencillo encontrar desarrolladores que conozcan tecnologías populares.
\item\underline{Librerías externas:} junto a la comunidad, las librerías externas ayudan a los desarrolladores a solucionar problemas que ya han sido resueltos
por otros desarrolladores. Por ejemplo, librerías para el manejo de fechas, librerías para gestionar conexiones con bases de datos...etc.
\item\underline{Licencias/mantenimiento de la tecnología:} es necesario tener en cuenta las licencias y el mantenimiento de las tecnologías antes de elegirlas,
pues pueden suponer gastos bastante elevados.
\end{itemize}

Teniendo en cuenta los apartados anteriores, se ha optado por utilizar Express.js junto sqlite para el desarrollo del HUB, e Ionic para el desarollo de
la interfaz gráfica.

\begin{figure}[H]
\centering
\includegraphics[width=5.00in]{images/stack_tecnologico.png}
\caption{Stack tecnológico escogido}
\label{fig:stack_teconologico}
\end{figure}

\subsection{Express.js}
Se trata de un framework para Node.js enfocado en el desarrollo de aplicaciones web. Utiliza JavaScript como lenguaje principal, un lenguaje moderno, muy popular y
muy rápido en ejecución. Los desarrolladores de express definen a su framework como: \textit{``Infraestructura web rápida, minimalista, y flexible para Node.js``}.
Además, según \textbf{hackr.io}: \textit{``Express es uno de los frameworks que más rapido está creciendo en popularidad, y es utilizado por compañías como Accenture, IBM o
Uber``}.
\par
Al ser un framework para Node, podemos utilizar librerías de terceros de manera sencilla a través del gestor de paquetes NPM. La comunidad de Node es una de las comunidades más activas
y grandes que existen.
\newpage
Según el \textbf{Developer Survey 2018} realizado por \textbf{StackOverFlow}, una de las comunidades de desarrolladores más grandes del mundo, JavaScript es la tecnología
 más popular (con un 71.5\%) dentro de las tecnologías de programación, scripting y lenguajes marcados:

\begin{figure}[H]
\centering
\includegraphics[width=5.00in]{images/tecnologias_populares.png}
\caption{Developer Surver Results 2018. Most popular programming, scripting and markup languagaes technologies.
 Recuperado de: https://insights.stackoverflow.com/survey/2018\#technology-\_-programming-scripting-and-markup-languages}
\label{fig:tecnologias_populares}
\end{figure}

\newpage
Además, el mismo estudio revela que Node.js es la tecnología más popular (con un 49.9\%) dentro del apartado de frameworks, librerías y herramientas:

\begin{figure}[H]
\centering
\includegraphics[width=6.00in]{images/frameworks_populares.png}
\caption{Developer Surver Results 2018. Most popular frameworks, libraries and tools. 
Recuperado de: https://insights.stackoverflow.com/survey/2018\#technology-\_-programming-scripting-and-markup-languages}
\label{fig:frameworks_populares}
\end{figure}

\lsection{Hub}
Como ya se ha explicado anteriormente, el HUB es la parte central de nuestro sistema. En él reside toda la lógica de negocio, y es el encargado
de comunicarse con la interfaz y los dispositivos.
\subsection{Tecnologías}
Para el desarrollo del HUB necesitamos utilizar tecnologías que nos permitan desarrollar un servidor REST de manera sencilla. Además, es conveniente
utilizar tecnologías modernas y mantenibles, y sobre todo, de ejecución ligera, ya que nuestro servidor se ejecutará idealmente en una raspberry.
\par
Teniendo en cuenta las necesidades tecnológicas
\subsection{Módulos}
En esta sección definiremos los módulos que nuestro HUB deberá tener y las funciones que cada uno debe cumplir. La organización por módulos, además
de permitirnos organizar nuestro software de una manera clara, nos ayudará a añadir nuevos módulos y funcionalidad el día de mañana con facilidad.
\par
Por ejemplo, en el módulo de servicios residirá toda la lógica interna, mientras que el módulo enrutador será el encargado de ``traducir`` los datos
provenientes de la red a un modelo de datos conocido e invocar a los diferentes servicios. De esta forma, si en un trabajo futuro queremos añadir
dispositivos bluetooth crearemos un módulo bluetooth que reutilice nuestros servicios.
\par
Otro ejemplo sería la migración de nuestra base de datos a otro motor diferente; sólo necesitaríamos cambiar el módulo repositorio, el resto del sistema
se mantendría intacto.
\par
Cada módulo
debe ser independiente del resto, y la modificación interna de un módulo no debería requerir la modificación del resto de módulos.
\subsection{Módulo enrutador}
Este módulo será el encargado de gestionar las conexiones entrantes y de manejar la información proveniente del exterior. Para nuestro caso, que utilizaremos
el protocolo HTTPS, en este módulo residirán las implementaciones de las APIS anteriormente definidas. Se encargará de implementar todas las rutas, encapsular
los diferentes parámetros en objetos de nuestro modelo y enviar las respuestas y códigos necesarios tras la invocación al módulo de servicios.
\subsection{Módulo middleware}
A pesar de haber separado este módulo del módulo enrutador, este módulo está totalmente ligado a la utilización del protocolo HTTPS. 
Se trata de un módulo totalmente independiente del módulo enrutador, y tendrá dos funciones principales:
\begin{itemize}
\setlength\itemsep{6pt plus 1pt minus 1pt}
\item Interceptar todas las peticiones antes de que lleguen al enrutador y validar las cabeceras y el token JWT. Si el token no es válido entonces
se envía un 401 Unauthorized sin llegar al enrutador.
\item Interceptar los errores que se provoquen durante la ejecución del programa (independientemente del módulo) y traducirlos a respuestas HTTPS. Para esto 
será necesario utilizar un modelo común de error que pueda ser interceptado por este módulo.
\end{itemize}
\subsection{Módulo de servicios}
En este módulo residirá la totalidad de nuestra lógica de negocio. A este módulo ya llegan objetos modelados con nuestro modelo de datos, y es totalmente
independiente del protocolo utilizado. Se encargará de hacer llamadas a los repositorios correspondientes y de aplicar la lógica correspondiente.
\par
Un ejemplo de lógica sería el registro de dispositivos; una vez recibido un dispositivo y sus correspondientes comandos, el servicio se encargará de hacer
las comprobaciones correspondientes y guardar el dispositivo y después sus comandos.
\par
Además, el módulo de servicios transformará los posibles errores provenientes de los repositorios para encapsularlos en errores internos. Un ejemplo sería
transformar un error \textbf{``14 SQLITE\_CANT\_OPEN``} en el siguiente error: \textbf{``Error 01: no se ha podido acceder a la base de datos sqlite``}.
\par
Tanto la entrada como la salida de datos de los métodos de nuestros servicios seguirán el modelo de datos del HUB.
\subsection{Módulo repositorio}
Este módulo contendrá toda la gestión de los datos del HUB. Será invocado por el módulo de servicios, y 
será el encargado de gestionar las conexiones con la base de datos e insertar/obtener datos de la misma.
Este módulo recibe datos modelados con nuestro modelo de datos, pero no necesariamente la manera de enviarlos/guardarlos tiene que coincidir con nuestro modelo 
de datos. Sin embargo, el retorno de los métodos de este módulo si serán datos modelados.
\par
Si en un futuro se realizasen llamadas a terceros, una API de Google por ejemplo, las llamadas a esas APIS también se realizarían desde este módulo.
\subsection{Vista general}
Por lo tanto, el diseño esquemático de la arquitectura interna del hub sería el siguiente:
\begin{figure}[H]
\centering
\includegraphics[width=6.00in]{images/arquitectura_hub.png}
\caption{Arquitectura interna del hub}
\label{fig:arquitectura_hub}
\end{figure}
\subsection{Configuración}
\subsection{Seguridad}
\lsection{Front end}
\subsection{Tecnologías}
\subsection{Componentes}
\subsection{Servicios}



\input{cap_experimentos}

\chapter{Conclusiones}
\label{chap:conclusiones}


\newpage \thispagestyle{empty} % Página vacía 

\chapter*{Glosario de acrónimos}
\addcontentsline{toc}{chapter}{Glosario de acrónimos}

\begin{itemize}
  
\item{\textbf{Sistema domótico}: Un sistema domótico es el conjunto de controladores y dispositvios que hacen posible la automatización del hogar.}
\item{\textbf{Dispositivo domótico}: Dispositivo que nos ayuda a la automatización del hogar actuando o recopilando información. Ejemplos: sensores de temperatura, relés, cámaras...etc. }
\item{\textbf{Bridge}: Dispositivo que nos ayuda a controlar y administar diferentes dispositivos domóticos. Los dispositivos se conectan al bridge, y el cliente interactua directamente a través de él. }
\item{\textbf{REST}: Arquitectura software que se apoya en el protocolo HTTP. Se utiliza en arquitecturas cliente-servidor. El cliente tiene operaciones básicas y predefinidas: GET, POST, PUT, DELETE... Y el sevidor responde a las peticiones con su correspondiente código HTTP.
Cada recurso del servidor es direccionable a través de su URI.}
\item{\textbf{JSON}: JavaScript Object Notation: formato de texto sencillo para el intercambio de datos.}
\item{\textbf{API}: Del inglés Application Programming Interface: conjunto de funciones y procedimientos que pueden ser utilizadas por otro software.}
\item{\textbf{Angular5}: Framework de código abierto mantenido por Google para la creación y mantenimiento de Single Page Applications (SPA). Desarrollado en TypeScript. }
\item{\textbf{SPA}: Del inglés Single Page Application: aplicación web que se ejecuta en una sola página, sin necesidad de refrescar el navegador, haciendo más fluida la navegación.}
\item{\textbf{MVC}: Modelo Vista Controlador: arquitectura software que separa los datos (Modelo) de la interfaz de usuario (Vista), su comunicación y lógica se encuentra en el controlador.}
\item{\textbf{Responsive}: Diseño web cuyo objetivo es adaptar la apariencia de la página web a diferentes dispositivos.}
\item{\textbf{Man-In-The-Middle}: Tipo de ataque en el que el atacante es capaz de interceptar y/o modificar mensajes enviados entre dos partes. 
Comúnmente este ataque se realiza en redes donde se utilizan protocolos HTTP, el atacante es capaz de captar las peticiones y modificarlas.}
\end{itemize}

\newpage \thispagestyle{empty} % Página vacía

\addcontentsline{toc}{chapter}{Bibliografa}    %Agregamos al ndice el capitulo de bibliografa 

\bibliographystyle{unsrt}   %plain pero ordenado en orden de aparacicion en documento principal
\bibliography{bibliografia}

\appendix   %Indicamos que lo que viene a continuación son apéndices

%\frontmatter %Para poner los anexos en numeros romanos

\input{anexo_manualuso}

\chapter{Manual del programador}


\newpage \thispagestyle{empty} % Página vacía 

%Hoja final en blanco
\newpage \thispagestyle{empty} % Página vacía

\end{document}
