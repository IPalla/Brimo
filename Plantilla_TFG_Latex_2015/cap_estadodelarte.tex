\chapter{Estado del arte}
\label{chap:estadodelarte}

\lsection{Introducción}
La domótica, \textbf{\textit{domus}} (casa en latín) \textbf{\textit{autónomo}} (autogobernado en griego), se refiere a todos los sistemas
que son capaces de automatizar un hogar (o cualquier otro edificio). La automatización del hogar nos puede ayudar con tareas del día a día,
y tienen multitud de aplicaciones. Explicaremos algunas aplicaciones prácticas de la domótica en la siguiente sección, y se hará un breve
resumen de la domótica en la actualidad para terminar con el capítulo.

\lsection{Aplicaciones}

La domótica tiene infinidad de aplicaciones que nos pueden ayudar en nuestro hogar. Los ámbitos principales en los que la domótica nos puede
ser de gran ayuda se encuentran:

\begin{itemize}
\item\underline{Seguridad}: la domótica nos puede ayudar a mantener nuestro hogar seguro de diferentes maneras. Las alarmas, por ejemplo, automatizan 
la acción de llamar a la policía si un intruso es detectado por un sensor. Además, las alarmas antifuego son capaces de detectar humo y hacer sonar 
sirenas e incluso ayudar a la extinción del fuego mediante la expulsión de agua.
\item\underline{Eficiencia energética}: los sistemas domóticos nos pueden facilitar el ahorro energético en nuestros hogares. Es una de las aplicaciones
más utilizadas, sobre todo en grandes edificios (hoteles, fábricas, hospitales...), ya que el ahorro energético supone un ahorro económico. Como ejemplo
de esta aplicación existen: luces encendidas a través de sensores de movimiento, termostatos automatizados que regulan la temperatura en función de la
temperatura ambiente y la franja horaria, enchufes inteligentes que se encienden o se apagan según las necesidades pre programadas.
\item\underline{Comodidad}: la domótica puede ahorrarnos mucho tiempo automatizando acciones cotidianas. Ejemplos de estas acciones son: apagado de luces,
programación de alarmas, control multimedia de los dispositivos, etc.
\item\underline{Accesibilidad}: esta aplicación de la domótica ayuda a las personas con limitaciones a tener más autonomía y seguridad en su día a día. Ya
existen dispositivos como la teleasistencia de la Cruz Roja \cite{CruzRoja:teleasistencia}, que permiten a las personas avisar a emergencias tan sólo pulsando un botón.
Además, existen sistemas de vigilancia remota, timbres lumínicos para personas con problemas de audición...etc.
\end{itemize}

\lsection{La domótica en la actualidad}

Actualmente existen multitud de dispositivos domóticos, fabricantes y protocolos de comunicación. Cerraduras inteligentes, cámaras de seguridad, altavoces
termostatos son algunos de los dispositivos que nos podemos encontrar en la actualidad.
\par
Existen instalaciones domóticas de tres tipos:
\begin{itemize}
\item\underline{Cableadas}: los dispositivos se comunican entre ellos por cable. Una instalación cableada requiere un gran coste, y en edificios antiguos 
son menos factibles.
\item\underline{Inalámbricas}: los dispositivos se comunican entre sí utilizando medios inalámbricos como WiFi, Bluetooth o infrarrojos.
\item\underline{Mixtas}: se trata de una instalación en la que existen tanto dispositivos cableados como dispositivos inalámbricos. Debido a la diversidad
de medios y protocolos utilizados en estas instalaciones, se requiere un ``bridge`` que gestione todas estas conexiones.
\end{itemize}

Además, existen diferentes arquitecturas de sistemas domóticos:
\begin{itemize}
\item\underline{Centralizados}: en estas arquitecturas existe un dispositivo central, también conocido como \textbf{bridge}, que es el encargado
de comunicarse con el resto de dispositivos: actuadores y sensores. La ventaja de estas arquitecturas es que pueden albergar dispositivos que utilicen
diferentes protocolos y medios físicos de comunicación (instalación mixta). Un inconveniente de este tipo de arquitectura es la necesidad del
 propio bridge, suele suponer costes elevados y la rotura o desconexión del dispositivo central supone la caída total del sistema.
\item\underline{Descentralizados}: en esta arquitectura existen varios dispositivos centrales. Los dispositivos se conectan a estos dispsoitivos centrales, 
y pueden existir diferentes tipos de dispositivos centrales. Estos dispositivos son más complejos, pero también son más ampliables, ya que puede que si existe
un número elevado de dispositivos la arquitectura centralizada puede colapsarse.
\item\underline{Distribuidos}: en esta arquitectura todos los dispositivos son a su vez controladores. Son capaces de enviar la información al sistema y actuar 
en consecuencia a información recibida de otros dispositivos. Estos sistemas son más fiables, ya que la desconexión de un dispositivo no significa 
la caída del sistema, pero son más complejos y requieren más programación.
\end{itemize}

