\chapter{Pruebas}
\label{chap:pruebas}

\lsection{Introducción}
\label{sect:intropruebas}
En este capítulo se describirán las pruebas realizadas para la verificación del correcto funcionamiento del sistema.

Antes de comenzar a describir las pruebas realizadas es necesario diferenciar entre diferentes tipos de tests:
\begin{itemize}
\item\underline{Tests unitarios}: este tipos de tests sólo prueban una clase o método del sistema. El resto de componentes del sistema se
maquetan para simular su comportamiento. Este tipo de tests verifica el correcto comportamiento de una parte del sistema, no del sistema
completo. Son muy interesantes para testear métodos complejos con mucha casuística.
\item\underline{Tests de integración}: este tipo de tests prueban la totalidad del sistema. A diferencia que en los tests unitarios, no se
maquetan elementos internos del sistema, se prueban todos los elementos juntos.
\end{itemize}

Durante la realización del proyecto se ha decidido hacer únicamente tests de integración, debido a los siguientes motivos:
\begin{itemize}
\item\underline{Nivel de verificación}: los tests unitarios por sí mismos no verifican el correcto funcionamiento del sistema, mientras que los tests de integración sí.
\item\underline{Mantenimiento}: los tests unitarios son muy costosos de mantener. Por lo general la refactorización de un método/clase, implica la 
modificación o realización de sus tests unitarios. Sin embargo, la refactorización de un método/clase no requiere cambiar los tests de integración. Además,
los tests de integración nos aseguran hacer una refactorización correcta, si al hacer una refactorización los tests de integración fallan, entonces hemos 
refactorizado mal. 

\end{itemize}

\lsection{Tests de integración (back-end)}
\label{sect:pruebasback}

Para los tests de integración del back-end hemos creado una colección de Postman, donde se prueba cada uno de los métodos de la API expuesta. Además, para 
comprobar el correcto funcionamiento de los accesos a base de datos se ha utilizado ``DB Browser for SQLite``.
\par
Una vez probado cada método individualmente se pasa el Collection Runner de Postman \ref{Postman:CollectionRunner}, que ejecuta todos los métodos que existen en una colección. Todos los métodos
de la colección llevan un test que verifica que el código de respuesta sea correcto, y se pueden añadir tests específicos para ese método.
\par
En nuestro caso se ha creado un flujo que lleva a cabo las siguientes acciones:

\begin{enumerate}
\item\underline{Login}: se hace login con el usuario por defecto y se guarda el token generado en la variable ``ACCESS_TOKEN``
 para usarse en las peticiones posteriores.
\item\underline{Nuevo usuario}: se añade un usuario nuevo utilizando como token el token anteriormente generado. El id del usuario creado se
guarda en la variable ``USER_ID``.
\item\underline{Modificación usuario}: se modifica el usuario creado, accediendo a él a través de su identificador previamente guardado.
\item\underline{Eliminación usuario}: se elimina el usuario creado, accediendo a él a través de su identificador previamente guardado.
\item\underline{Añadir dispositivo}: se añade un nuevo dispositivo y se guarda su identificador en la variable ``DEVICE_ID``.
\item\underline{Actualizar info dispositivo}: se actualiza la información del dispositivo utilizando su identificador. Se envía como información
la palabra ``updatedinfo``.
\item\underline{Listar dispositivos}: se obtiene el listado de dispositivos. El primer dispositivo de la lista debe tener la palabra ``updatedinfo``
en el campo información.
\item\underline{Obtener información dispositivo}: se obtiene la información del dispositivo a través de su identificador. Esta información debe contener la información actualizada.
\item\underline{Añadir localización nueva}: se añade una nueva localización y se salva su identificador en la variable ``ROOM_ID``.
\item\underline{Editar localización}: se edita la localización añadida a través de su identificador, y se cambia su nombre a ``new_room_updated``.
\item\underline{Listar localizaciones}: se obtiene el listado de localizaciones, que debe contener la localización añadida y editada anteriormente.
\item\underline{Editar dispositivo}: se edita la localización del dispositivo añadido con la nueva localización.
\item\underline{Eliminar dispositivo}: se elimina el dispositivo.
\item\underline{Eliminar localización}: se elimina la localización añadida anteriormente.
\item\underline{Enviar comando a dispositivo}: se envía un comando al dispositivo, este test debe fallar debido a que no existe ningún dispositivo real que 
recepcione el comando.
\end{enumerate}

Antes de pasar los tests de integración se debe eliminar la base de datos, y al finalizar la ejecución de los tests la base de datos debe quedar vacía, a
excepción del usuario que se crea por defecto.

\lsection{Tests de integración completos}
\label{sect:intecompletos}
 
