\chapter{Pruebas}
\label{chap:pruebas}

\lsection{Introducción}
En este capítulo se describirán las pruebas realizadas para la verificación del correcto funcionamiento del sistema.

Antes de comenzar a describir las pruebas realizadas es necesario diferenciar entre diferentes tipos de tests:
\begin{itemize}
\item\underline{Tests unitarios}: este tipos de tests sólo prueban una clase o método del sistema. El resto de componentes del sistema se
maquetan para simular su comportamiento. Este tipo de tests verifica el correcto comportamiento de una parte del sistema, no del sistema
completo. Son muy interesantes para testear métodos complejos con mucha casuística.
\item\underline{Tests de integración}: este tipo de tests prueban la totalidad del sistema. A diferencia que en los tests unitarios, no se
maquetan elementos internos del sistema, se prueban todos los elementos juntos.
\end{itemize}

Durante la realización del proyecto se ha decidido hacer únicamente tests de integración, debido a los siguientes motivos:
\begin{itemize}
\item\underline{Nivel de verificación}: los tests unitarios por sí mismos no verifican el correcto funcionamiento del sistema, mientras que los tests de integración sí.
\item\underline{Mantenimiento}: los tests unitarios son muy costosos de mantener. Por lo general la refactorización de un método/clase, implica la 
modificación o realización de sus tests unitarios. Sin embargo, la refactorización de un método/clase no requiere cambiar los tests de integración. Además,
los tests de integración nos aseguran hacer una refactorización correcta, si al hacer una refactorización los tests de integración fallan, entonces hemos 
refactorizado mal. 

\end{itemize}

\lsection{Tests de integración (back-end)}
\label{sect:sistemasreferencia}

Para los tests de integración del back-end hemos creado una colección de Postman, donde se prueba cada uno de los métodos de la API expuesta. Además, para 
comprobar el correcto funcionamiento de los accesos a base de datos se ha utilizado ``DB Browser for SQLite``.


Se han hecho pruebas forzando el error, 

\lsection{Tests de integración completos}
\label{sect:escenarios_pruebas}
 
