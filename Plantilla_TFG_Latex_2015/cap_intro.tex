\chapter{Introducción} 
\label{chap:intro}

\vspace{-0.2cm}

\lsection{Motivación del proyecto}

Los sistemas domóticos por lo general utilizan una arquitectura centralizada: un controlador (bridge) es el encargado de enviar y recibir información de los dispositivos domóticos y las interfaces.
 Se utilizan sistemas centralizados debido a que abaratan mucho el coste de los dispositivos domóticos, así los dispositivos tienen poca electrónica y programación, y la responsabilidad principal
 reside en el bridge. Este enfoque tiene sentido cuando se trata de muchos dispositivos en un hogar, que es el caso ideal, si solo tuviésemos un sensor carecería de sentido tener un sensor y un bridge para manejarlo.

El problema principal que existe con los sistemas centralizados se encuentra en la \textbf{compatibilidad} entre dispositivos y bridges.
Por lo que he observado~\cite{article:EstadoDelArte}, todavía falta mucha estandarización en el
ámbito de la domótica: cada fabricante usa sus medios y protocolos haciendo incompatibles bridges y dispositivos. Además, estos dispositivos no suelen ser 
muy asequibles. Por lo tanto, nos encontramos ante la necesidad de comprar todos los dispositivos de una misma marca o tener muchos bridges, lo que nos obligaría a manejar cada
dispositivo desde su correspondiente bridge.

La domótica puede hacernos la vida en el hogar mucho más sencilla, ayudándonos a ahorrar tiempo y dinero que podremos invertir en otras cosas. Los hogares todavía están muy poco automatizados, y mi principal motivación ha sido acercar la domótica 
a las personas y aprender acerca de ella. Gracias a nuestro sistema manejamos todos los dispositvos a través de un solo bridge de manera sencilla y eficaz.

\newpage
\lsection{Objetivos y enfoque}

El objetivo último de nuestro proyecto es desarrollar un sistema que sea capaz de recibir y enviar información de dispositivos domóticos y sea capaz de interactuar con el cliente. 

Los \textbf{requisitos} que debe cumplir nuestro sistema son:
\begin{itemize}
  \item \underline{Ligero.} Un bridge no debería necesitar demasiada capacidad de procesamiento y de memoria, y es necesario que no sea muy costoso, por lo tanto, la ligereza es requisito indispensable.
  \item \underline{Compatibilidad.} Necesitamos que nuestro bridge no sea únicamente compatible con un tipo de sensor, o un modelo de cámara
  \item \underline{Interfaz sencilla y adaptable a cualquier dispositivo.} Necesitamos que la interfaz de nuestro bridge sea compatible con cualquier dispositivo sin perder funcinalidad.
  \item \underline{Seguridad.} La seguridad en la domótica es algo indispensable, confío en que el día de mañana incluso las cerraduras de nuestras casas serán automáticas, y no podemos dejar la responsabilidad de la seguridad
  de nuestra a casa a un sistema con vulnerabilidades de seguridad.
  \item \underline{Escalable.} Nuestro sistema ha de ser escalable y debemos pensar en todo momento en ampliaciones y trabajos futuros. La domótica evoluciona a pasos agigantados y podríamos añadir funcionalidades a nuestro 
  sistema practicamente a diario. No obstante, es necesario acotar firmemente los límites de nuestro proyecto para ceñirnos a las horas que corresponden a un TFG, aunque debemos tener muy en cuenta en todo momento trabajos futuros y ampliaciones.
  Además, debemos tener en cuenta la escalabilidad: domótica en un hospital, en una ciudad...etc.
\end{itemize}

\lsection{Metodología y plan de trabajo}

Otro ejemplo de imagen:
\begin{figure}[h]
  \centerline{
    \mbox{\includegraphics[width=3.00in]{images/logo_uam.eps}}
  }
  \caption{Ejemplo pie de figura 2}
  \label{fig:norm_Daugman}
\end{figure}

\newpage \thispagestyle{empty} % Página vacía 