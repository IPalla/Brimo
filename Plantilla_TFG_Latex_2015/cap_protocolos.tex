\chapter{Protocolo}
\label{chap:protocolos}

Este capítulo esta dedicado a describir los protocolos utilizados en este proyecto, apoyados sobre HTTP e implementados con un servidor CherryPy.
El back-end de nuestro bridge va a diferenciar entre dos tipos de comunicación: comunicación directa con los dispositivos y comunicación con el front-end de la interfaz.

\lsection {Características y necesidades del protocolo}


\lsection{Comunicación con los dispositivos} \label{sec:ProtocolosDispositivos}

En esta sección se describirá la comunicación con los dispositivos: registro de dispositivos y actualización de la información. El sentido de la comunicación es dispositivo -> bridge, y todos los tipos de dispositivos 
siguen este protocolo.

\begin{table}[h]
    \centering
    \scriptsize
    \begin{tabular}{|l|l|l|l|}
    \hline
        MÉTODO & ENDPOINT    & DESCRIPCION                                                & SEGURIDAD \\ \hline
        POST    & /devices/register    & Registra un nuevo dispositivo.            & NO     \\ \hline
        PUT    & /device/:id & Edita la información de un dispositivo.                              & NO     \\ \hline
    \end{tabular}
\end{table}

\subsection{Registro de nuevos dispositivos} \label{sec:RegistroDispositivos}

La primera vez que un dispositivo se conecta al bridge es necesario que realice una primera llamada de registro. De esta manera, el bridge dará de alta el dispositivo
y sabrá de qué tipo de dispositivo se trata.

